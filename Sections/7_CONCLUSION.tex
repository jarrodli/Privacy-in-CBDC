\section{Conclusion}

Privacy in CBDC has established a sound foundation by composing well-understood cryptographic primitives into strategies that strengthen an individual's economic privacy. This paper explains how two different exploratory CBDC projects have utilised different technical strategies to enhance both identity and transaction privacy. It is shown that the frameworks through which these projects implement CBDC vary both technically and conceptually. By contrasting these solutions against the backdrop of the government's inherent interest in the anonymity of domestic transactions, this paper recognises the different avenues open to a central bank in balancing an individual's privacy against other privacy eroding objectives including regulatory enforcement. While it is clear there exist both advantages and disadvantages to the adoption of different frameworks, it is shown that these are invariably a product of both technical limitations and social need.
