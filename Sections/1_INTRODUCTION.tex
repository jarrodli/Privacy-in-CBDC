\section{Introduction}

In recent years, individual privacy has often given way to the proliferation of technology in everyday life. No clearer is this demonstrated than in the case of Australia's COVIDSafe app [1] which provided a means by which the Australian government could trace the activities of a COVID positive individual to find others at risk of infection. A specific risk with the debasement of financial data privacy is that it may cause objective consumer detriment [2], in the sense that consumers are unable to make informed choices about the goods and services they purchase. Societal expectations of privacy have thus assumed greater importance in the creation of new technologies, especially where these technologies are applied to areas of significant public importance, such as the supply and distribution of currency.

This paper will consider the extent to which privacy is protected in the creation of digital currencies. In so doing, it begins in Part II by exploring the role of the central bank, particularly in relation to the creation of a Central Bank Digital Currency. Through an assessment of prior work, Part III explains a number of historically important advances towards the preservation of privacy in the context of payment systems. Part IV introduces the first case study for analysis, an Australian exploratory proposal called Project Atom. Part V introduces the second case study, an ongoing Swedish digital currency project called the Riskbank e-krona. With these two case studies in mind, part VI contrasts and compares the technical and, by proxy, regulatory aspects of the projects. Part VII concludes.
