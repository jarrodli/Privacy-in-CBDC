\section{Case Study: Australia's Project Atom}

\subsection{Syndicated Lending Platform}

A \textit{syndicated loan} [24] is a type of structured debt that allows two or more creditors (i.e., a syndicate of creditors) to provide liquidity to a borrower. The syndicated loan market is specialised in the sense that it services market participants seeking large amounts of money. Syndicated loans take the shape of a single agreement between multiple parties, with the loan administered by a single creditor. These ``arrangers'' or ``lead managers'' [25] are typically large retail banks with access to a central bank reserves accounts. The syndicated loan market may therefore be characterised as a wholesale market, requiring the provision of a wholesale CBDC.

Project Atom [26] is a proof-of-concept platform that provisions wholesale CBDC to syndicated lenders. Under such a system, (1) an ESA holder requests an amount of CBDC from the central bank, (2) the central bank issues the CBDC to the ESA holder, (3) an amount equal to that of the CBDC is transferred into a consolidated fund via. RITS, (4) the retail bank distributes the CBDC to third parties, and (5) the third party spends the CBDC by ``burning'' the token and claiming an amount equal to the CBDC from the consolidated fund.

\subsection{AZTEC Cryptographic Engine}

The AZTEC Cryptographic Engine (`AZTEC') is the primary means adopted by Project Atom to satisfy the functional requirements of privacy and anonymity [27]. AZTEC uses a \textit{UXTO note model} to represent ownership and token value within the network. It also gives parties the option to use \textit{stealth addresses} to hide a recipient's identity. To carry out transactions, the project adopts a protocol called \textit{join-split}. Finally, the settlement of a transaction can be contingent upon proof that the ownership of a note meets some regulatory threshold, defined as a \textit{range proof}.

\textit{UXTO note model}. An AZTEC note [27] is an ``encrypted representation of an abstract value''. It comprises two public and two private pieces of information. The private variables of a note include (1) an encrypted value of the note called an AZTEC commitment, and (2) the Ethereum address of the owner. The public variables include (3) the actual value of the note, and (4) the note's decryption key called a viewing key. The state of all notes is managed by a \textit{note registry}. Given that the value of each note is encrypted, this model preserves transaction privacy.

\textit{Stealth addresses}. While normal Ethereum addresses may be used to identify owners of an AZTEC note, the use of a single address allows the correlation of transactions and poses the risk of identity disambiguation. To remedy this, AZTEC provides recipients with the option to hide their Ethereum address through the publishing of a stealth address $Q$ [27]-[28] formed by two public-private key pairs $(a, A), (b, B)$. The spender uses $(A, B)$ to sign the transaction before broadcasting it to the network. Conversely, the recipient verifies and spends transactions with $(a, b)$. Crucially, $Q$ is not associated with the recipient's identity and, therefore, offers identity privacy.

\textit{Join-split transactions}. A join-split operation [27] is defined by taking one or more notes from the note registry (join), combining these notes together, and dividing the sum into one or more output notes (split). Formally, if there exist a set of notes $x$ consisting of $n$ inputs and $m$ outputs, then $\sum_{i=0}^{n} x_i  = \sum_{j=0}^{m} x_j$. After a join-split operation takes place, the $n$ input notes are deemed ``burned'' and the output notes are available in the note registry for spending.

\textit{Range proofs}. AZTEC recognises the need for running computations on notes to prove the existence of certain attributes. For instance, a regulatory requirement that transactions not exceed a publicly known threshold could be imposed on every join-split operation. AZTEC uses concepts from zero knowledge arguments, homomorphic encryption and multi-party computation to privately prove that a transaction does not exceed this threshold [27]. 

A ``trusted setup phase'' begins by generating a publicly known set of elliptic curve points $[a, b]$. A value $v$ is taken to lie under a regulatory threshold, or alternatively within a particular \textit{range}, if $v \in [a, b)$. Following this, the prover produces a commitment $c(v)$ where $c$ provides homomorphic addition as a function of an elliptic curve. Finally, the well-formedness of $v$ is determined by diffusing $c(v)$ across multiple parties that co-jointly prove the commitment falls within $[a,b)$, as discussed in the previous section. Again, so long as one party $p_i$ destroys $c(v)_i$, it is not possible to reconstruct $c(v)$.
