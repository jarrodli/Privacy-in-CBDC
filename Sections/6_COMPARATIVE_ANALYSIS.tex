\section{Comparative Analysis}

\subsection{Privacy Model}

At its core, AZTEC is predicated on the assumption that all parties interested in the transactions within a network prefer complete privacy and anonymity. In nearly all but the most exceptional cases, the cryptographic protocols adopted satisfy this requirement. Those exceptional cases include finding an efficient solution to the discrete logarithm problem and the possibility that all parties are untrustworthy during the multi-party computation procedure. Notwithstanding these exceptional cases, AZTEC can provably guarantee the privacy of transaction data, including transaction value.

By contrast, Corda applies the concept of partial visibility to provide a limited form of transaction anonymity. As discussed, the unavoidable consequence is that notary nodes are privy to any and all transactions within a network [36]. This means that there is \textit{at least} one entity with access to the entire transaction graph, including transaction value and party identities. Similarly, the notion that an oracle is given \textit{some} information about a transaction runs counter to the notion of transaction privacy advanced by AZTEC.

With respect to identity privacy, both AZTEC and Corda provide the option to completely anonymise identities. The schemes used by these frameworks are relatively similar. That is, they both yield an ephemeral public key to settle transactions that do not reveal the recipient's true identity.

AZTEC's UTXO note model is computationally secure from information leakage. However, there exist methods of inferring transaction knowledge notwithstanding this underlying premise. For instance, join-split transactions are susceptible to knowledge inference. That is, if one note of value $10$ is used to transact with another party for an asset of value $5$, the note is split into two notes of value $5$, with one note provided to the transferee. The transferee may then infer that the remaining value of the second note must be of value $5$ less than its original value. Thus, in certain circumstances, \textit{some} information can be leaked about AZTEC transactions.

Conversely, Corda provides no guarantee against information leakage. In fact, Corda explicitly leaks information to every party required to execute a transaction. Therefore, Corda does not maintain the privacy of transaction data.

\subsection{Regulatory Compliance}

At the core of both AZTEC and Corda is the problem posed at the start of this paper. Without restating that question, there exist two limbs to the issue. First, every government is interested in the extent to which a platform might provide a layer of privacy to transactions that finance illicit activities, particularly terrorism and money laundering. And, second, every individual (whether a natural person or corporate entity) is interested in the extent to which a platform might provide a consolidated ledger of their spending habits and, by proxy, a window into day-to-day life.

As is now clear, AZTEC preserves individuals' privacy to the greatest degree, while Corda does so to a lesser extent. On such a cursory analysis, Corda is of major appeal to governments seeking a degree of transparency in the execution of transactions. Yet it remains important to realise that, in the context of AZTEC's range proofs, zero-knowledge does not mean zero-verification. Range proofs allow regulators to verify that every transaction satisfies legislated requirements in a manner that does not intrude into an individual's privacy. Ultimately, both platforms provide regulators with methods of enforcing stringent compliance standards.

One significant aspect of Corda is its innate ability to harness the field of RegTech [37]-[38]. That is, the use of technology to enhance regulatory and compliance processes. Where unusual transactions are carried out, their legitimacy might be called into question. For instance, there could exist parties who are regularly transferred unusually large amounts of money that, while under the regulatory threshold, sum to an extremely large amount. By providing a means for regulators to take notice of these suspicious transactions, Corda enhances, rather than detracts from, primary regulatory objectives.

Regulators might adopt an approach that allows anonymity for a fixed number of low-value transactions, but forbids anonymity for high-value or high-volume transactions. This concept is not new and can be traced to a foundational principle of the e-CNY: ``small amounts are anonymous, big amounts are traceable'' [39]. The upshot of this principle is the protection of individual privacy, especially of those who may otherwise have transacted in physical cash and coins. In either case, the suitability of AZTEC or Corda for building a CBDC depends by and large on the method of implementation [37].
